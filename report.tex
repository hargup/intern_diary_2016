\protect\hypertarget{anchor}{}{}Technical Alternative to Encrypted Media
Extensions (EME)

\protect\hypertarget{anchor-1}{}{}Introduction

Encrypted Media Extensions (EME) are a draft specification\footnote{Encrypted
  Media Extensions W3C Candidate Recommendation
  \href{https://www.w3.org/TR/encrypted-media/}{\emph{https}}\href{https://www.w3.org/TR/encrypted-media/}{\emph{://}}\href{https://www.w3.org/TR/encrypted-media/}{\emph{www}}\href{https://www.w3.org/TR/encrypted-media/}{\emph{.}}\href{https://www.w3.org/TR/encrypted-media/}{\emph{w}}\href{https://www.w3.org/TR/encrypted-media/}{\emph{3.}}\href{https://www.w3.org/TR/encrypted-media/}{\emph{org}}\href{https://www.w3.org/TR/encrypted-media/}{\emph{/}}\href{https://www.w3.org/TR/encrypted-media/}{\emph{TR}}\href{https://www.w3.org/TR/encrypted-media/}{\emph{/}}\href{https://www.w3.org/TR/encrypted-media/}{\emph{encrypted}}\href{https://www.w3.org/TR/encrypted-media/}{\emph{-}}\href{https://www.w3.org/TR/encrypted-media/}{\emph{media}}\href{https://www.w3.org/TR/encrypted-media/}{\emph{/}}.
  For a general overview see https://hsivonen.fi/eme/} to standardize
digital rights management (DRM) for audio and video at the browser
level.

The specification has been very controversial in the software community
since it was first drafted in 2012. It was proposed by content providers
and streaming service operators to ensure that content delivered to
legitimate users is inaccessible to pirates.

However, the proposed solution raised salient questions about
interoperability, privacy, accessibility and implementation in Free and
Open Source (FOSS) software.

Several parties have, over the course of the discussion at W3C, proposed
several alternate technical alternatives. This report aims to analyze
these alternatives and the proposed EME specification along six
dimensions; technical copy protection, legal copy protection,
interoperability/entry barriers for browsers, privacy, accessibility,
and user security.

\protect\hypertarget{anchor-2}{}{}Aims of the Specification

\begin{itemize}
\tightlist
\item
  Make it technically hard for a malicious user to pirate a particular
  media
\item
  Have sufficient legal barriers to deter infringement
\end{itemize}

At the same time:

\begin{itemize}
\tightlist
\item
  Ensure interoperability and make sure there are no entry barriers for
  new browsers
\item
  Protect privacy of users
\item
  Make sure the system doesn't bring about security vulnerability
\item
  Maintain accessibility for a person with disabilities
\end{itemize}

\protect\hypertarget{anchor-3}{}{}Metrics of Comparison

\protect\hypertarget{anchor-4}{}{}Technological Copy Protection :

During the transfer of video content from the web server of the content
provide to the user, there are multiple points where a malicious entity
can capture the copyrighted content.

We classify the technical strength of a DRM system depending on the
point in transition where the capture can take place. Assuming the
server is itself secure, the first point where the adversary can capture
the media is during the transition from the server to the user's device.
Preventing such kind of interception is a standard problem and is in
solved by the use of HTTPS. After the media stream reaches the device of
the intended user, she can capture the before it is played on the media
software. For example in case of images or text, the user can usually
save the media without the need of any special software or specialized
technique. So the next step from content providers side is build
restrictions in the software playing the media. The usual way to do this
is by making sure that the media can be played only on certain software
which doesn't allow the user to copy the media. The software
restrictions can be implemented using arbitrary codecs, scrambling or
encryption. Technical restrictions at software level are always prone to
be captured by screen capturing softwares, and hardware emulators which
appears as output devices to media software but are used to save the
media instead. To prevent capturing at software level there exists
technologies such as HDCP\footnote{HDCP Whitepaper,
  \href{https://web.archive.org/web/20080920191718/http://www.digital-cp.com/files/documents/04A897FD-FEF1-0EEE-CDBB649127F79525/HDCP_deciphered_070808.pdf}{\emph{https}}\href{https://web.archive.org/web/20080920191718/http://www.digital-cp.com/files/documents/04A897FD-FEF1-0EEE-CDBB649127F79525/HDCP_deciphered_070808.pdf}{\emph{://}}\href{https://web.archive.org/web/20080920191718/http://www.digital-cp.com/files/documents/04A897FD-FEF1-0EEE-CDBB649127F79525/HDCP_deciphered_070808.pdf}{\emph{web}}\href{https://web.archive.org/web/20080920191718/http://www.digital-cp.com/files/documents/04A897FD-FEF1-0EEE-CDBB649127F79525/HDCP_deciphered_070808.pdf}{\emph{.}}\href{https://web.archive.org/web/20080920191718/http://www.digital-cp.com/files/documents/04A897FD-FEF1-0EEE-CDBB649127F79525/HDCP_deciphered_070808.pdf}{\emph{archive}}\href{https://web.archive.org/web/20080920191718/http://www.digital-cp.com/files/documents/04A897FD-FEF1-0EEE-CDBB649127F79525/HDCP_deciphered_070808.pdf}{\emph{.}}\href{https://web.archive.org/web/20080920191718/http://www.digital-cp.com/files/documents/04A897FD-FEF1-0EEE-CDBB649127F79525/HDCP_deciphered_070808.pdf}{\emph{org}}\href{https://web.archive.org/web/20080920191718/http://www.digital-cp.com/files/documents/04A897FD-FEF1-0EEE-CDBB649127F79525/HDCP_deciphered_070808.pdf}{\emph{/}}\href{https://web.archive.org/web/20080920191718/http://www.digital-cp.com/files/documents/04A897FD-FEF1-0EEE-CDBB649127F79525/HDCP_deciphered_070808.pdf}{\emph{web}}\href{https://web.archive.org/web/20080920191718/http://www.digital-cp.com/files/documents/04A897FD-FEF1-0EEE-CDBB649127F79525/HDCP_deciphered_070808.pdf}{\emph{/20080920191718/}}\href{https://web.archive.org/web/20080920191718/http://www.digital-cp.com/files/documents/04A897FD-FEF1-0EEE-CDBB649127F79525/HDCP_deciphered_070808.pdf}{\emph{http}}\href{https://web.archive.org/web/20080920191718/http://www.digital-cp.com/files/documents/04A897FD-FEF1-0EEE-CDBB649127F79525/HDCP_deciphered_070808.pdf}{\emph{://}}\href{https://web.archive.org/web/20080920191718/http://www.digital-cp.com/files/documents/04A897FD-FEF1-0EEE-CDBB649127F79525/HDCP_deciphered_070808.pdf}{\emph{www}}\href{https://web.archive.org/web/20080920191718/http://www.digital-cp.com/files/documents/04A897FD-FEF1-0EEE-CDBB649127F79525/HDCP_deciphered_070808.pdf}{\emph{.}}\href{https://web.archive.org/web/20080920191718/http://www.digital-cp.com/files/documents/04A897FD-FEF1-0EEE-CDBB649127F79525/HDCP_deciphered_070808.pdf}{\emph{digital}}\href{https://web.archive.org/web/20080920191718/http://www.digital-cp.com/files/documents/04A897FD-FEF1-0EEE-CDBB649127F79525/HDCP_deciphered_070808.pdf}{\emph{-}}\href{https://web.archive.org/web/20080920191718/http://www.digital-cp.com/files/documents/04A897FD-FEF1-0EEE-CDBB649127F79525/HDCP_deciphered_070808.pdf}{\emph{cp}}\href{https://web.archive.org/web/20080920191718/http://www.digital-cp.com/files/documents/04A897FD-FEF1-0EEE-CDBB649127F79525/HDCP_deciphered_070808.pdf}{\emph{.}}\href{https://web.archive.org/web/20080920191718/http://www.digital-cp.com/files/documents/04A897FD-FEF1-0EEE-CDBB649127F79525/HDCP_deciphered_070808.pdf}{\emph{com}}\href{https://web.archive.org/web/20080920191718/http://www.digital-cp.com/files/documents/04A897FD-FEF1-0EEE-CDBB649127F79525/HDCP_deciphered_070808.pdf}{\emph{/}}\href{https://web.archive.org/web/20080920191718/http://www.digital-cp.com/files/documents/04A897FD-FEF1-0EEE-CDBB649127F79525/HDCP_deciphered_070808.pdf}{\emph{files}}\href{https://web.archive.org/web/20080920191718/http://www.digital-cp.com/files/documents/04A897FD-FEF1-0EEE-CDBB649127F79525/HDCP_deciphered_070808.pdf}{\emph{/}}\href{https://web.archive.org/web/20080920191718/http://www.digital-cp.com/files/documents/04A897FD-FEF1-0EEE-CDBB649127F79525/HDCP_deciphered_070808.pdf}{\emph{documents}}\href{https://web.archive.org/web/20080920191718/http://www.digital-cp.com/files/documents/04A897FD-FEF1-0EEE-CDBB649127F79525/HDCP_deciphered_070808.pdf}{\emph{/04}}\href{https://web.archive.org/web/20080920191718/http://www.digital-cp.com/files/documents/04A897FD-FEF1-0EEE-CDBB649127F79525/HDCP_deciphered_070808.pdf}{\emph{A}}\href{https://web.archive.org/web/20080920191718/http://www.digital-cp.com/files/documents/04A897FD-FEF1-0EEE-CDBB649127F79525/HDCP_deciphered_070808.pdf}{\emph{897}}\href{https://web.archive.org/web/20080920191718/http://www.digital-cp.com/files/documents/04A897FD-FEF1-0EEE-CDBB649127F79525/HDCP_deciphered_070808.pdf}{\emph{FD}}\href{https://web.archive.org/web/20080920191718/http://www.digital-cp.com/files/documents/04A897FD-FEF1-0EEE-CDBB649127F79525/HDCP_deciphered_070808.pdf}{\emph{-}}\href{https://web.archive.org/web/20080920191718/http://www.digital-cp.com/files/documents/04A897FD-FEF1-0EEE-CDBB649127F79525/HDCP_deciphered_070808.pdf}{\emph{FEF}}\href{https://web.archive.org/web/20080920191718/http://www.digital-cp.com/files/documents/04A897FD-FEF1-0EEE-CDBB649127F79525/HDCP_deciphered_070808.pdf}{\emph{1-0}}\href{https://web.archive.org/web/20080920191718/http://www.digital-cp.com/files/documents/04A897FD-FEF1-0EEE-CDBB649127F79525/HDCP_deciphered_070808.pdf}{\emph{EEE}}\href{https://web.archive.org/web/20080920191718/http://www.digital-cp.com/files/documents/04A897FD-FEF1-0EEE-CDBB649127F79525/HDCP_deciphered_070808.pdf}{\emph{-}}\href{https://web.archive.org/web/20080920191718/http://www.digital-cp.com/files/documents/04A897FD-FEF1-0EEE-CDBB649127F79525/HDCP_deciphered_070808.pdf}{\emph{CDBB}}\href{https://web.archive.org/web/20080920191718/http://www.digital-cp.com/files/documents/04A897FD-FEF1-0EEE-CDBB649127F79525/HDCP_deciphered_070808.pdf}{\emph{649127}}\href{https://web.archive.org/web/20080920191718/http://www.digital-cp.com/files/documents/04A897FD-FEF1-0EEE-CDBB649127F79525/HDCP_deciphered_070808.pdf}{\emph{F}}\href{https://web.archive.org/web/20080920191718/http://www.digital-cp.com/files/documents/04A897FD-FEF1-0EEE-CDBB649127F79525/HDCP_deciphered_070808.pdf}{\emph{79525/}}\href{https://web.archive.org/web/20080920191718/http://www.digital-cp.com/files/documents/04A897FD-FEF1-0EEE-CDBB649127F79525/HDCP_deciphered_070808.pdf}{\emph{HDCP}}\href{https://web.archive.org/web/20080920191718/http://www.digital-cp.com/files/documents/04A897FD-FEF1-0EEE-CDBB649127F79525/HDCP_deciphered_070808.pdf}{\emph{\_}}\href{https://web.archive.org/web/20080920191718/http://www.digital-cp.com/files/documents/04A897FD-FEF1-0EEE-CDBB649127F79525/HDCP_deciphered_070808.pdf}{\emph{deciphered}}\href{https://web.archive.org/web/20080920191718/http://www.digital-cp.com/files/documents/04A897FD-FEF1-0EEE-CDBB649127F79525/HDCP_deciphered_070808.pdf}{\emph{\_070808.}}\href{https://web.archive.org/web/20080920191718/http://www.digital-cp.com/files/documents/04A897FD-FEF1-0EEE-CDBB649127F79525/HDCP_deciphered_070808.pdf}{\emph{pdf}}}
which protects the media during its transition from the media software
to the output device. Although such technologies are also fallible to a
user holding a video camera in front of the monitor. This weakness of
the DRM systems is known as Analog Hole.

Technological Copy Protection is:

\begin{itemize}
\tightlist
\item
  High: Infringer needs specialized hardware to capture the copyrighted
  content.
\item
  Medium: Infringer needs specialized Software to capture the
  copyrighted content
\item
  Low: Infringer needs only commonly available software and hardware to
  capture the copyrighted content
\end{itemize}

\protect\hypertarget{anchor-5}{}{}Copy Protection (Legal)

Jurisdictions across the world have laws which make it illegal to
circumvent technological protections methods for the protections of
Copyright. The most famous of them is the Section 1201 of the United
States Digital Millennium Copyright Act (DMCA). For content providers
who wish to use TPMs to prevent piracy of their copyrighted work, these
laws provide additional layers of protection. DMCA disallows
circumventing a technical measure which effectively control access to
copyrighted work, also it disallows the ``manufacture, import, offer to
the public, provide, or otherwise traffic in any technology, product,
service, device, component'' which is primarily designed to circumvent a
DRM. DMCA has an exception that allows the reverse engineering of a DRM
when solely done to provide interoperability.\footnote{Section 1201, US
  Digital Millennium Copyright Act }

Legal protection against infringement is high in DRM system if:

\begin{itemize}
\tightlist
\item
  High: Circumventing the DRM and creating tools to enable that is
  illegal unconditionally
\item
  Medium: Circumventing the DRM and creating tools to enable that is
  illegal depending the intent and circumstances
\item
  Low: Circumventing the DRM and creating tools to enable that is legal.
\end{itemize}

\protect\hypertarget{anchor-6}{}{}Security

DRM systems have been criticized for leaving users' devices vulnerable.

Security of a user using the DRM system is:

\begin{itemize}
\tightlist
\item
  High: The system don't require any elevated permissions
\item
  Medium: The system only requires elevated software permissions 
\item
  Low: The system requires both elevated hardware or software permission
\end{itemize}

\protect\hypertarget{anchor-7}{}{}Privacy

Privacy of user using the DRM systems is:

\begin{itemize}
\tightlist
\item
  High: The system doesn't collect minimal information
\item
  Medium: The system only collects non personally identifiable
  information
\item
  Low: The system collects personally identifiable information
\end{itemize}

\protect\hypertarget{anchor-8}{}{}Accessibility

DRM systems can turn out to be problematic for providing accessibility
for disabled persons. In case of video service can be made accessible by
providing access to closed captions for a video and by modifying the
stream to make it accessible to color blind people. However, a DRM
system could present unnecessary barriers for people trying to provide
accessibility solutions. There can be technical barrier in the process
of handling the video stream

W3C Technical Architecture Group (TAG) suggested following guidelines to
maintain accessibility in Encrypted Media Extensions:\footnote{https://github.com/w3ctag/eme/blob/master/EME\%20Proposal.md\#accessibility-1}

\begin{itemize}
\tightlist
\item
  ensuring that media content may be redirected to certain system
  services;
\item
  ensuring that every piece of digital content is available in its
  original form (for example, subtitles are not blended into video,
  etc);
\item
  ensuring that standard operations (adjusting contrast, using
  third-party subtitles or audio-stream) may be applied to restricted
  media;
\item
  ensuring that restricted media from different sources provided by
  different EME systems (for example, video from one source and
  sign-language interpretation of that video from another source) may be
  used simultaneously.
\end{itemize}

We say accessibility in a DRM system is:

\begin{itemize}
\tightlist
\item
  High: If all the of the guidelines are met
\item
  Medium: If two more points in the guideline are met
\item
  Low: If less than points of the guideline are met
\end{itemize}

\protect\hypertarget{anchor-9}{}{}Interoperability

Interoperability of any system is important to keep the entry barriers
low for a new producer to enter the market. Interoperability of a DRM
system for browsers is:

\begin{itemize}
\tightlist
\item
  High: The full spec is available for implementation on royalty free
  basis
\item
  Medium: The full spec is not available, but can be implemented through
  reverse engineering without legal barriers.
\item
  Low: Third parties may restrict new browsers from implementing the
  spec through legal means.
\end{itemize}

\protect\hypertarget{anchor-10}{}{}Specifications

\protect\hypertarget{anchor-11}{}{}EME Specification

EME specification only defines the javascript component of the system
and the large component called Content Decryption Module(CDM) is left
undefined. The CDM can be hardware based using technologies like HDCP,
which prevents screen capture. The CDM can be software based and can
return the decrypted video to the browser to render, or it can use its
own media stream and render it by itself. Most of the CDMs in use are
proprietary but there can exist CDMs which are fully specified and are
open source. The implications for copy protection, privacy,
accessibility and security depends on the CDM used. Interoperability of
EME spec is very low because there are not only technical barriers due
lack of full specification but also legal barriers as browsers may need
to get into a contract with the dominant CDM providers to add support
for their CDM.

\protect\hypertarget{anchor-12}{}{}Obfuscation (Arbitrary Codec)

Charles Pritchard pointed out the HTML5 video specification is codec
agnostic, hence the content providers can stream the media using an
arbitrary codec which only supported by the media provider.\footnote{https://lists.w3.org/Archives/Public/public-html/2012Feb/0328.html}
So even if the user captures the video stream it cannot be pirated
without reverse engineering the codec. Although reverse engineering is
usually allowed by DRM laws hence the legal protection is low.\footnote{Section
  1201, Digital Millennium Copyright Act} Since the codec support is
provided through OS, there is no need to modify the browser and the
system can be supported by any browser without any technical or legal
barriers.

\protect\hypertarget{anchor-13}{}{}HTTPS and JS encryption

Tab Atkins proposed using JS encryption using browser and
\textless{}video\textgreater{} element\footnote{https://lists.w3.org/Archives/Public/public-html/2012Feb/0456.html}.
Since the technique requires the a malicious user to implement the full
\textless{}video\textgreater{} spec to decrypt the video, the scheme
provides moderate technical copy protection.

\protect\hypertarget{anchor-14}{}{}Encryption using video tag

According to David Singer encrypted video can be played through the
existing \textless{}video\textgreater{} tags where the content file says
its content-ID and is marked as protected, someone who has the DRM to
play the content installed and has brought the keys to play it can watch
the video.\footnote{https://lists.w3.org/Archives/Public/public-html/2012Feb/0422.html}
As a concrete example he talked about protected .m4p audio files from
iTunes library, which plays just fine on Safari.\footnote{https://lists.w3.org/Archives/Public/public-html/2012Feb/0433.html}

\protect\hypertarget{anchor-15}{}{}Plugin System (Flash)

Existing plugin system, mainly Flash is be used to as a technical
measure to prevent copyright infringement. It is more interoperable than
EME because any browser with a correct implementation of NPAPI can
provide support for Flash\footnote{https://lists.w3.org/Archives/Public/public-html/2012Feb/0427.html}.

\begin{longtable}[]{@{}@{}}
\toprule
\tabularnewline
\tabularnewline
\tabularnewline
\tabularnewline
\tabularnewline
\tabularnewline
\bottomrule
\end{longtable}

\protect\hypertarget{anchor-16}{}{}Diversity Analysis

We considered that any email on the public-html mailing list of W3C with
`` EME ``, ``Encrypted Media'' or ``Digital Rights Management'' in the
subject line is about the Encrypted Media Extensions draft
specification. Then we manually tagged every participant by their
gender, the region they belong to and their stakeholder community.

\protect\hypertarget{anchor-17}{}{}Region

\begin{longtable}[]{@{}@{}}
\toprule
\tabularnewline
\tabularnewline
\tabularnewline
\tabularnewline
\tabularnewline
\tabularnewline
\tabularnewline
\tabularnewline
\bottomrule
\end{longtable}

We found that there were no absolutely participants from Asia, Africa or
South America.

The Internet is lived differently in different parts of the world. The
IP laws in many countries in the global South are very different to
those in the USA or Europe. In addition, many internet users in these
countries use connections with relatively low bandwidths. The lack of
representation of people from the global South means that their concerns
-\/- technical, cultural, and legal -\/- are not being considered at all
in this debate.

\protect\hypertarget{anchor-18}{}{}

\protect\hypertarget{anchor-19}{}{}Stakeholder Community

\begin{longtable}[]{@{}@{}}
\toprule
\tabularnewline
\tabularnewline
\tabularnewline
\tabularnewline
\tabularnewline
\tabularnewline
\tabularnewline
\tabularnewline
\tabularnewline
\tabularnewline
\bottomrule
\end{longtable}

We observe that there was no participation from the Security Researcher
community and negligible participation from privacy community. Voice of
Digital Content Provider was overrepresented with almost 40\% of emails
sent by them.

Methodological remarks:

\begin{itemize}
\tightlist
\item
  Participants are categorized on the basis stakes of their employer and
  not specifically on the work they do. For example someone who works on
  privacy in Google will be placed in "DRM platform provider" instead of
  "Privacy".
\item
  W3C and Universities are considered to neutral and their employees are
  categorized by the work they do.
\item
  Google's position is very interesting, it is a DRM provider as a
  browser manufacturer but also a content provider in Youtube and fair
  number of Google Employers are against EME due to other concerns.
  Therefore Christian Kaiser has been paced as Content provider because
  he works on Youtube, and everyone else has been placed as DRM
  provider.
\end{itemize}

\protect\hypertarget{anchor-20}{}{}Gender

\begin{longtable}[]{@{}@{}}
\toprule
\tabularnewline
\tabularnewline
\tabularnewline
\tabularnewline
\bottomrule
\end{longtable}

There was only one women participating in the discussing contributing
1.3 \% of the emails sent. The numbers reflects widely discussed lack of
gender diversity in Tech and Open communities\footnote{http://geekfeminism.wikia.com/wiki/FLOSS}.
