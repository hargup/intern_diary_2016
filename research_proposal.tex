\documentclass{article}

\title{Research Proposal: Stakeholder Analysis on Encrypted Media Enhancement}
\author{Harsh Gupta}
\begin{document}
\maketitle

\section*{Introduction}

The arrival of internet and digital technologies have made it is very easy and
cheap to create perfect copies of digital media. This has produced serious
difficulties to produce and distribute media using the pre internet era
business models. To continue with their old ways media companies have tried to
make it hard for people to make copies of digital media and working with
hardware manufacturers and software platform providers they have various types
of copy restrictions in hardware and software that is used to play the media.
These technologies are largely known as Digital Rights Management or Digital
Restriction Management or simply DRM.

DRMs have remained controversial since their inception, by media companies DRM
is seen as something essential to preserve their business models and continue
operating profitably whereas many digital rights activist see DRM as something
fundamentally opposed to user's ability to own and control their devices.

Modern DRM usually works by storing and transmitting the media in encrypted
formats which is usually decrypted only while the playing or rending of the
media on end user's devices. On the world wide web the existing way to play and
transmitted encrypted media content is through third party plugins like Adobe's
Flash Player and Microsoft silverlight. There is an ongoing draft proposal by
W3C to standardize this in HTML5 with a standard called Encrypted Media
Enhancements(EME) \cite{david_dorwin_encrypted_????}. EME too has remained
controversial since 2013 when the work on the standard started.

\section*{Literature Review}

DRM, its laws, economics and technology encompasses a lots of different areas and have
been covered fairly well by the existing academic literature. Review of the
literature points out	

\begin{itemize}

\item The media companies have potential benefit from having strict DRM protection. \cite{varian2005copying}

\item The providers of the software and hardware platform also benefit greatly from
  DRM because of lock-ins. \cite{varian2007drm}

\item DRM technologies are usually not perfect people find a way around it. And the
  analogue hole aways exist. %\cite[Chapter~22]{anderson2008security}

\item Prevalence of DRM fairly threatens many legitimate uses of media provided under
  the Copyright Law, thus also hampering amateur culture. \cite{eff_unintended_2013}

\item Some of the DRM technologies have proved to be security risks for end
users. \cite{anderson2008security} %\cite[Chapter~22]{anderson2008security}

\item DRM technology risks decreasing the accessibility of digital content.
\cite{cory_doctorow_interoperability_2016}

\item DRM laws have been used beyond its intended aim of restricting copying of
  copyrighted work to limit interoperability and hamper free speech.
\cite{eff_unintended_2013}

\end{itemize}

Several blog posts both in support and opposition to EME exist.

\begin{itemize}

\item The arguments for EME points to the fact that DRM in web is not something
new and they present EME as largely an effort to standardize it. \cite{w3c_information_????}

\item The argument against EME points to the harms of DRM and to the
  expected lack of interoperability between different implementations. \cite{cory_doctorow_interoperability_2016}


\end{itemize}

A lot of primary data on EME is also available in the form of mailing
discussion, discussions on the github issues tracker and pull requests where
the draft is being formulated.  None of the secondary literature looks at the
process and historical context through which W3C has reached the current form
of the draft. Also there is insufficient literature on DRM and EME with focus
on Indian stakeholder.

The web being very pervasive, the implications of any technical standard
proposed by W3C goes beyond technical details, the literature review revealed
no literature focusing on the functioning of W3C as a standard setting
organization.


\section*{Research Objective}

Our research goals are two fold,

\begin{enumerate}

\item Do a stakeholder analysis of the proposed EME standard, understanding and
  explaining the various pros and cons of EME standard for various stakeholders
involved, especially focusing on the Indian media industry and Indian public in
general.

\item In the process of answering the first question analyze W3C as a standard setting
   organization and also find out what is the process through which an
individual or organization can participate in W3C and can get their voices
heard.

\end{enumerate}


\section*{Methodology}

The process will mainly include reading and analyzing the data available
through W3C mailing list archives and the discussions on EME's github
repository. To find out the affiliations of the participants we will have to
look at their online profiles that includes personal websites and linked
profiles.

To understand EME in Indian context we'll mainly use the secondary literature
on the nature, extent of illegal copying and the recent empirical studies than
have been done. If necessary we might also conduct interviews of the
stakeholders involved.

\section*{Output}

\begin{itemize}
\item A report/paper on the stakeholder analysis.
\item A handbook on how to participate at W3C
\end{itemize}


\section*{Timeline}

\begin{itemize}
\item 1 month
\end{itemize}

2 weeks to study the literature available, and another 2 weeks to analyze and
compile the analysis in a report.


\section*{Budget}

\begin{itemize}
\item A small amount of budget might be required for the procurement of books and
  research material.
\end{itemize}

\bibliographystyle{aaai}
\bibliography{research_proposal}
\end{document}
